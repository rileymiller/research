% This is annote.bib
% Author: Ayman Ammoura
% A demo for CMPUT 603 Fall 2002.
% The order of the following entries is irrelevant. They will be sorted according to the
% bibliography style used.
%%%%%%%%%%%%%%%%%%%%%%%%%%%%%%%%%%%%%%%%%%%%%%%%%%%%%%%%%%%%%%%%%%%%%%%%%%%%%%%%%%%%%%%%%

% The following is a conference paper
@article{lovins1968development,
  title={Development of a stemming algorithm},
  author={Lovins, Julie Beth},
  journal={Mech. Translat. \& Comp. Linguistics},
  volume={11},
  number={1-2},
  pages={22--31},
  year={1968}
}

@inproceedings{silva2003importance,
  title={The importance of stop word removal on recall values in text categorization},
  author={Silva, Catarina and Ribeiro, Bernardete},
  booktitle={Proceedings of the International Joint Conference on Neural Networks, 2003.},
  volume={3},
  pages={1661--1666},
  year={2003},
  organization={IEEE}
}

@inproceedings{kanis2010comparison,
  title={Comparison of different lemmatization approaches through the means of information retrieval performance},
  author={Kanis, Jakub and Skorkovsk{\'a}, Lucie},
  booktitle={International Conference on Text, Speech and Dialogue},
  pages={93--100},
  year={2010},
  organization={Springer}
}

@article{jivani2011comparative,
  title={A comparative study of stemming algorithms},
  author={Jivani, Anjali Ganesh and others},
  journal={Int. J. Comp. Tech. Appl},
  volume={2},
  number={6},
  pages={1930--1938},
  year={2011}
}

@inproceedings{larkey2002improving,
  title={Improving stemming for Arabic information retrieval: light stemming and co-occurrence analysis},
  author={Larkey, Leah S and Ballesteros, Lisa and Connell, Margaret E},
  booktitle={Proceedings of the 25th annual international ACM SIGIR conference on Research and development in information retrieval},
  pages={275--282},
  year={2002}
}

@article{schofieldunderstanding,
  title={Understanding Text Pre-Processing for Latent Dirichlet Allocation},
  author={Schofield, Alexandra and Magnusson, M{\aa}ns and Thompson, Laure and Mimno, David}
}

@article{boyd2014care,
  title={Care and feeding of topic models: Problems, diagnostics, and improvements},
  author={Boyd-Graber, Jordan and Mimno, David and Newman, David},
  journal={Handbook of mixed membership models and their applications},
  volume={225255},
  year={2014},
  publisher={CRC Press Boca Raton, FL}
}

@article{blei2003latent,
  title={Latent dirichlet allocation},
  author={Blei, David M and Ng, Andrew Y and Jordan, Michael I},
  journal={Journal of machine Learning research},
  volume={3},
  number={Jan},
  pages={993--1022},
  year={2003}
}

@article{dieng2019topic,
  title={Topic modeling in embedding spaces},
  author={Dieng, Adji B and Ruiz, Francisco JR and Blei, David M},
  journal={arXiv preprint arXiv:1907.04907},
  year={2019}
}

@article{loper2002nltk,
  title={NLTK: the natural language toolkit},
  author={Loper, Edward and Bird, Steven},
  journal={arXiv preprint cs/0205028},
  year={2002}
}

@inproceedings{wang2007topical,
  title={Topical n-grams: Phrase and topic discovery, with an application to information retrieval},
  author={Wang, Xuerui and McCallum, Andrew and Wei, Xing},
  booktitle={Seventh IEEE international conference on data mining (ICDM 2007)},
  pages={697--702},
  year={2007},
  organization={IEEE}
}

@article{moody2016mixing,
  title={Mixing dirichlet topic models and word embeddings to make lda2vec},
  author={Moody, Christopher E},
  journal={arXiv preprint arXiv:1605.02019},
  year={2016}
}

@article{richardson2007beautiful,
  title={Beautiful soup documentation},
  author={Richardson, Leonard},
  journal={April},
  year={2007}
}

@inproceedings{Sint2009CombiningUF,
  title={Combining Unstructured, Fully Structured and Semi-Structured Information in Semantic Wikis},
  author={Rolf Sint and Stephanie Stroka and Sebastian Schaffert and Roland Ferstl},
  booktitle={SemWiki},
  year={2009}
}
@article{papadimitriou2000latent,
  title={Latent semantic indexing: A probabilistic analysis},
  author={Papadimitriou, Christos H and Raghavan, Prabhakar and Tamaki, Hisao and Vempala, Santosh},
  journal={Journal of Computer and System Sciences},
  volume={61},
  number={2},
  pages={217--235},
  year={2000},
  publisher={Elsevier}
}

@software{gensim,
title={gensim},
medium={Online},
available={https://radimrehurek.com/gensim/}
}

@article{rehurek2011,
  added-at = {2018-08-13T00:00:00.000+0200},
  author = {Rehurek, Radim},
  biburl = {https://www.bibsonomy.org/bibtex/279e686466735408be365ea22b622beb2/dblp},
  ee = {http://arxiv.org/abs/1102.5597},
  interhash = {018cc6da76f65159ae8066529ab6fdad},
  intrahash = {79e686466735408be365ea22b622beb2},
  journal = {CoRR},
  keywords = {dblp},
  timestamp = {2018-08-14T14:54:39.000+0200},
  title = {Fast and Faster: A Comparison of Two Streamed Matrix Decomposition Algorithms},
  url = {http://dblp.uni-trier.de/db/journals/corr/corr1102.html#abs-1102-5597},
  volume = {abs/1102.5597},
  year = 2011
}

@inproceedings{attias2000variational,
  title={A variational baysian framework for graphical models},
  author={Attias, Hagai},
  booktitle={Advances in neural information processing systems},
  pages={209--215},
  year={2000}
}

@inproceedings{hoffman2010online,
  title={Online learning for latent dirichlet allocation},
  author={Hoffman, Matthew and Bach, Francis R and Blei, David M},
  booktitle={advances in neural information processing systems},
  pages={856--864},
  year={2010}
}

@article{deerwester1990indexing,
  title={Indexing by latent semantic analysis},
  author={Deerwester, Scott and Dumais, Susan T and Furnas, George W and Landauer, Thomas K and Harshman, Richard},
  journal={Journal of the American society for information science},
  volume={41},
  number={6},
  pages={391--407},
  year={1990},
  publisher={Wiley Online Library}
}

@inproceedings{hofmann1999probabilistic,
  title={Probabilistic latent semantic indexing},
  author={Hofmann, Thomas},
  booktitle={Proceedings of the 22nd annual international ACM SIGIR conference on Research and development in information retrieval},
  pages={50--57},
  year={1999}
}

@techreport{wang2005note,
  title={A note on topical n-grams},
  author={Wang, Xuerui and McCallum, Andrew},
  year={2005},
  institution={MASSACHUSETTS UNIV AMHERST DEPT OF COMPUTER SCIENCE}
}

@inproceedings{mikolov2013distributed,
  title={Distributed representations of words and phrases and their compositionality},
  author={Mikolov, Tomas and Sutskever, Ilya and Chen, Kai and Corrado, Greg S and Dean, Jeff},
  booktitle={Advances in neural information processing systems},
  pages={3111--3119},
  year={2013}
}

@inproceedings{Salton1968AutomaticIO,
  title={Automatic information organization and retrieval mcgraw-hill (1968)},
  author={Gerard Salton},
  year={1968}
}

@inproceedings{bouayad1999duplication,
  title={Duplication in corpora},
  author={Bouayad-Agha, Nadjet and Kilgarriff, Adam},
  booktitle={Proceedings of the Second CLUK Colloquium},
  year={1999}
}
%% ACM conference paper
@InProceedings{Cheng2016,
  title = {Wide \& Deep Learning for Recommender Systems},
  author = {H. Cheng L. Koc J. Harmsen T. Shaked T. Chandra H. Aradhye G. Anderson G. Corrado W. Chai M. Ispir R. Anil Z. Haque L. Hong V. Jain X. Liu, H. Shah},
  booktitle = {Proceedings of the 1st Workshop on Deep Learning for Recommender Systems},
  year = {2016},
  month = {September},
  annote = {Cheng et al. implemented a combonation of wide and deep learning to surface relevant content to 
  users on the Google Play store. They used the two different techniques to combat some of the model fallacies that
  can appear in recommender systems. The team of researchers tackled the two primary tradeoffs for recommender systems 
  by using wide and deep learning to optimize for memorization and generalization. Memorization in recommender
  systems "can be defined as learning the frequent concurrence of items and features and exploiting the correlation in historical
  data". On the other hand, generalization is based on transitivity of correlation and explores new and rare feature combination, this
  approach leads to greater diversity of recommendations. On a practical basis, the researchers utilized wide learning to help with the memorization
  aspect, many recommender systems utilize simple logistic regression algorithms to surface relevant content. However, this approach falls short when
  trying to query item feature pairs that haven't appeared in the training data. On the other hand, many researchers have tried to use
  deep learning neural networks to generate recommendations since they're able to generate feature pairs that haven't appeared in the training set,
  however, these deep learning approaches can over generalize and make less relevant reccomendations. These researchers made use of joint training
  which simultaneously optimzes the parameters of the wide and deep models at training time. The teams data pipeline consisted of three stages: data generation, 
  model training, and model serving. During training their model input layer took in training data and vocabulary and generated sparse and dense features with labels.
  The wide learning component consisted of the cross product transformation of user installed apps and impression apps. The deep learning model consisted of a 32 dimension
  embedding vector which is learned for each categorical feature. These embeddings are then concatenated together with dense features, resulting in a dense vector of
  approximately 1200 dimensions. These models were trained on over 500 billion examples. To optimize for performance, the researchers optimized the performance by using multithreading
  parallelism by running smaller batches in parallel instead of running all of the candidate inferences through the model at the same time. The recommender servers recommend over 10 million
  apps a second.
  https://dl.acm.org/citation.cfm?id=2988454
  }
}

%% Classification in crowd sourcing platforms
@article{sun2014chimera,
  title={Chimera: Large-scale classification using machine learning, rules, and crowdsourcing},
  author={Sun, Chong and Rampalli, Narasimhan and Yang, Frank and Doan, AnHai},
  journal={Proceedings of the VLDB Endowment},
  volume={7},
  number={13},
  pages={1529--1540},
  year={2014},
  publisher={VLDB Endowment},
  annote = {
    In this paper the team at Walmart Labs developed a tool referred to as Chimera which implements a combination of learning, rules, and crowdsourcing to classify products by their
    product description. Their solution demonstrates that classical assumptions that are applied to classification problems break down at the scale that they deal with (~5000+ categories)
    and that a combination of approaches is required to solve the problem.
  }
}

%% Quality Control in crowdsourcing systems
@article{allahbakhsh2013quality,
  title={Quality control in crowdsourcing systems: Issues and directions},
  author={Allahbakhsh, Mohammad and Benatallah, Boualem and Ignjatovic, Aleksandar and Motahari-Nezhad, Hamid Reza and Bertino, Elisa and Dustdar, Schahram},
  journal={IEEE Internet Computing},
  volume={17},
  number={2},
  pages={76--81},
  year={2013},
  publisher={IEEE},
  annote={
  This paper focuses on quality control in crowdsourcing platforms like Amazon MTURK focusing on ensuring high quality crowd contributions. The authors introduce a taxonomy for crowdsourcing platforms
  that records the workers' reputation and expertise. One of the major issues with quality in crowdsourcing platforms is how subjective quality can be. The researchers outline a potential solution for improving
  quality-control approaches by developing a machine-learning based recommender system that will surface relevant quality control approaches to requestors based on the requester's profile, history, as well as the history
  of the crowd and the quality requirements of the task. One of the other topics of potential research is the development of crowdsource platform middleware which will solve the problem of having distributed crowds over various
  crowdsourcing platforms. This would be an interesting area for a machine learning application.
  }
}

%% Deep Learning in recommendation systems
@inproceedings{elkahky2015multi,
  title={A multi-view deep learning approach for cross domain user modeling in recommendation systems},
  author={Elkahky, Ali Mamdouh and Song, Yang and He, Xiaodong},
  booktitle={Proceedings of the 24th International Conference on World Wide Web},
  pages={278--288},
  year={2015},
  organization={International World Wide Web Conferences Steering Committee},
  annote={
  TODO read and annotate
  }
}

%% Data Shapley, quantifying value of data
@inproceedings{Ghorbani2019DataSE,
  title={Data Shapley: Equitable Valuation of Data for Machine Learning},
  author={Amirata Ghorbani and James Y. Zou},
  booktitle={ICML},
  year={2019},
  annote={
    In this paper the authors applied Shapley techniques from game theory in economics to quantify the value of a data point in a data set used to train a model. This researcher
    is interesting because it shows quantitative performance improvement of existing machine learning models applied to an existing data set. It was really interesting how the authors
    were able to apply this concept from game theory to evaluate the contribution of a data point in training data on the model data and the corresponding actions of removing poisoning data
    or add similar data points was able to considerably increase or decrease the performance of machine learning techniques. This research shines light on two very interesting concepts, the first
    being how to improve model effectiveness by quantitating the contribution of each data point in the training set--the second being how to evaluate the value of user data to compensate user's for
    their data. This concept may be used in my research to evaluate the contribution of users or tasks in my training data sets which can be removed or lead to the addition of valuable data based on
    similar data to improve my algorithm performance.
  }
}

%% Survey of Machine Learning in Recommender Systems
@article{portugal2018use,
  title={The use of machine learning algorithms in recommender systems: A systematic review},
  author={Portugal, Ivens and Alencar, Paulo and Cowan, Donald},
  journal={Expert Systems with Applications},
  volume={97},
  pages={205--227},
  year={2018},
  publisher={Elsevier},
  annote = {
  This study researches many of the primary machine learning algorithms used in real reccommender systems (RS). The purpose of their study was to provide researchers a comprehensive overview
  of the work being done on RSs and potential areas for further research, specifically targeting the identification of ML algorithms that are used in RSs and to discover open questions in RS development that might be impacted
  by software engineering. The authors outline that there are three main categories of recommender systems: collaborative, content-based, and hybrid filtering.
  The collaborative approach pertains to considering user data when processing information for recommendations (recommendations based off other users). The content-based approach has to do with item data the
  user can access. The hybrid approach combines both collaborative and content based filtering. Many applications collect both explicit and implicit information about the user and store it on the
  user's computer using the browser's local storage, accessed using the browser's API. The explicit data that is collected occurs when users are aware that they are providing information
  about themselves, whereas, during implicit data collection user's are often unaware that their data is being collected (user history on the site, clicks, keystrokes, etc.). The most commonly used algorithms included Bayesian,
  Decision Tree, Matrix factorization-based, Neighbor-based, Neural Networks, and rule learning. The authors theorized that Bayesian and Decision Tree algorithms appeared most commonly because they are less complex than some of
  the other approaches. The authors suggested that there is a lot of work that could be done in researching the application of neural networks to RS.
  }
}
  
%% Preference Learning for Reccomender Systems
@article{de2009preference,
  title={Preference learning in recommender systems},
  author={De Gemmis, Marco and Iaquinta, Leo and Lops, Pasquale and Musto, Cataldo and Narducci, Fedelucio and Semeraro, Giovanni},
  journal={Preference Learning},
  volume={41},
  pages={41--55},
  year={2009},
  annote={
    The authors outlines the need for reccomender systems to surface relevant content in the internet of information overload. The authors begin by describing the need to collect implicit and explicit user data to develop contextual
    recommendation systems. The basic architecture for recommender systems outlined in this paper includes a user profile, an algorithm to update the profile given usage/input information, and an adaptice tool that exploits the profile
    in order to provide personalization. One of the primary methodologies the authors pointed out was Collaborative Filtering which they defined as system's recommendation based on evaluation from users who share similar interests among
    them (may apply to my research based on worker and requestor ratings based on historical performance).
  }
}

%% Article on applying CF and deep learning to reccomender systems.
@ARTICLE{Wang2015,
  author = {H. Wang N. Wang and D.-Y Yeung},
  title = {Collaborative deep learning for recommender systems},
  journal = {Proc. KDD},
  pages = {1235-1244},
  year = {2015},
  annote = "These researchers introduced a concept they called collaborative deep learning (CDL) which is a hierarchical Bayesian model which combines deep representation learning for the content
  information and collaborative filtering for the ratings. The paper introduces three main approaches for recommender systems: content based models, collaborative filtering models, and hybrid methods.
  The content based methods make use of user profiles and product descriptions to make recommendations. CF approaches use history such as past activities or preferences without using user or product info. While
  hybrid methods utilize a mixture of both content and CF approaches. Limitations for content models are user privacy while CF approaches stumble when ratings on products are sparse. The researchers highlight that deep
  learning models are state of the art in other fields such as computer vision and NLP and that they are able to learn features automatically, yet they are inferior to shallow models such as CF when it comes to learning similarity and
  relationships between items. To construct their CDL model, the researchers utilized Stacked Denoising Autoencoders (SDAE) which is a feedforward  neural network for learning representations."
}

%%


% Deep Learning with Python Francois Chollet
@book{DeepLearningWithPython:book_typical,
  author        = "Francois Chollet",
  title         = "Deep Learning With Python",
  publisher     = "Manning",
  address       = "Shelter Island, NY",
  year          = "2018",
  annote        = "Deep learning is a mathematical framework for learning representations from data. The author describes deep learning as a multistage way too learn data representations.
  Deep learning has become widely popular because it automates the most crucial step in the machine learning workflow, \emph{feature engineering}. The automation of this process has simplified
  many complex data pipelines used for \emph{shallow learning} algorithms by learning all of the features in one end-to-end pass through a deep learning model. The author describes the process by which
  deep learning learns as \emph{the incremental, layer-by-layer way in which increasingly complex representations are developed and the fact that these intermediate representations are learned jointly}.
  One of the biggest breakthroughs that deep learning introduces is hte ability for a model to learn all layers of representation \emph{jointly}, at the same time, instead of in succession like the approach used
  for \emph{feature engineering} in shallow models. The author outlines that the two most effective techniques in Machine Learning are \emph{gradient boosting} for shallow learning where structured data is available and 
  deep learning for perceptual problems. Machine Learning is finally accessible due to advances in the internet, providing much larger data sets, and graphics chips that were developed for gaming. The gaming market effectively
  subsidized supercomputing for the next wave of machine learning, specifically deep learning.
  \\
  \\ \textbf{Definitions}
  \\ \textbf{Weights:} The specification of what a layer does to it's input data is stored in the layer's \emph{weights}.
  \\ \textbf{Parameterized:} The transformation implemented by a layer is \emph{parameterized} by its weights.
  \\ \textbf{Learning:} \emph{Learning} means finding a set of values for the weights of all layers in a networkm such that the network will correctly map example inputs to their associated targets.
  \\ \textbf{Loss function:} The \emph{loss function} in the context of Deep Learning is the method used to measure the output of a neural network and measure how far this output is from what was expected.
  \\ \textbf{Optimizer:} Uses \emph{backpropagation} to make the adjustment of weights based off the score of the \emph{loss function} in a manner that will lower the loss score for the current example.
  \\ \textbf{Backpropagation:} The central algorithm in deep learning. TODO add more here
  \\ \textbf{Training Loop:} At the beginning of training, the network weight's are assigned random values, thus, the network performs a series of random transformations and the loss score is correspondingly high.
  Then with every example the neural network processes the weights are adjusted a little each time in the correct direction, causing the loss score to decrease.
  \\ \textbf{Probabilistic Modeling:} One of the earliest forms of machine learning, Naive Bayes is an example. A closely related model is the \emph{logistic regression}, a classification algorithm. Logist regression is
  usually one of the first algorithms a data scientist will try on a data set to get a handle on the dataset.
  \\ \textbf{Support Vector Machines (SVM):} aim to solve classification problems by finding good \emph{decision boundaries} between two sets of points belonging to two different categories. SVMs find these boundaries in two steps: mapping data to a
  new high-dimensional representation where the \emph{decision boundary} can be expressed as a hyperplane and trying to maximize the distance between the hyperplane and the closest data points from each class in a process called
  \emph{maximizing the margin}. SVMs use the \emph{kernel trick} to find good decision hyperplanes in the new representation space. This happens by only computing the distance between the pairs of points in that space, which can be done using a \emph{kernel function}.
  A \emph{kernel function} is a computationally tractable operatoin that maps any two points in the initial space to the distance between these points in the targer representation space, bypassing the explicit computation of the new representation. SVMs were widely popular
  for a long time but were hard to scale to large data sets and perceptual problems. Since SVMs are a flavor of \emph{shallow learning}, applying SVMs to perceptual problems like image classification requires extracting useful representations manually (\emph{feature engineering}) which
  is difficult and brittle.
  \\ \textbf{Decision Trees:} A flowchart-like structure that enables the classification of input data points or to predict output values given inputs. Preferred to kernel methods by 2010.
  \\ \textbf{Random Forest:} An algorithm based on \emph{decision trees} which builds a large number of specialized decision tres and then ensembles their outputs. Almost always the second-best algorith for shallow machine-learning tasks.
  \\ \textbf{Gradient Boosting Machines:} Considered one of, if not \emph{the} best shallow learning algorithm. Gradient boosting machines is a technique based on ensembling weak prediction models, generally \emph{decision trees}. It uses \emph{gradient boosting} to improve any machine-learning model by iteratively training new models that specialize in addressing the weak points
  of the previous models. When applied to \emph{decision trees} the resulting models outperform \emph{decision trees} most of the time.
  \\ \textbf{CNN:} Deep convolutional neural networks, \emph{convnets}, became the goto algorithm for computer vision problems after becoming the goto technique for the ImageNet competition. Has completely replaced SVMs and decision tress in a wide range of applications.
  \\ \textbf{Overfitting:} When Machine learning models perform worse on new data than on their training data. TODO, add more
  \\ \textbf{Tensor:} The primitive datastructure for machine learning. A generalization of matrices to an arbitrary number of dimensions.
  \\ \textbf{Rank:} The number of dimensions of a matrix, in the context of ML the number of dimensions of a tensor. \emph{Dimensions} are often called an \emph{axis} in the context of tensors.
  "
}

@article{linden2003amazon,
  title={Amazon. com recommendations: Item-to-item collaborative filtering},
  author={Linden, Greg and Smith, Brent and York, Jeremy},
  journal={IEEE Internet computing},
  number={1},
  pages={76--80},
  year={2003},
  publisher={IEEE},
  annote= {One of the primary drawbacks of collaborative filtering is that it requires online computation and the algorithm performance is linear with regards to the number of customers in the dataset, this isn't scalable for massive retailers like Amazon unless you
   were to reduce the data sample or the dimensionality of the data set all of which would reduce the quality of the recommendations. An alternative to collaborative filtering is clustering models which perform offline calculations with the primary drawback being reduced quality
    of recommendations. The approach Amazon took was item-item collaborative filtering where instead of developing recommendations off of similar customer decisions the researchers keyed in only on surfacing items similar to items the customer had either bought or reviewed, drastically 
    reducing the computation and memory concerns of traditional collaborative filtering while still surfacing high quality recommendations.}
}

@inproceedings{okura2017embedding,
  title={Embedding-based news recommendation for millions of users},
  author={Okura, Shumpei and Tagami, Yukihiro and Ono, Shingo and Tajima, Akira},
  booktitle={Proceedings of the 23rd ACM SIGKDD International Conference on Knowledge Discovery and Data Mining},
  pages={1933--1942},
  year={2017},
  organization={ACM},
  annote = {In this paper the authors utilize a three step process to surface news articles. In the first step of the process they utilize a denoising autencoder, a RNN, and inner product operations to recommend relevant news information quickly.
    The authors used cosine similarity to reduce duplication of articles before they submit data to the input layer of the RNN, this allows them to reduce the processing time by reducing the level of useless input. This would be an interesting technique that could be applied to reducing duplicate tasks.
    They also used offline experiments using logs recorded on their server to determine the accuracy of their RNN. The researchers training set was massive, recording over one billion browses. The test set was reduced to a random sample of 500,000 sessions.
    The researchers compared a GRU (Gated Recurrent Unit) RNN to a variety of techniques and found the GRU RNN to perform best. However, after left in production for over three months the RNN performance had degraded to that of the words based model approach, due in part 
    that it took over a week to train the RNN.
  }
}

@inproceedings{wang2017dynamic,
  title={Dynamic attention deep model for article recommendation by learning human editors' demonstration},
  author={Wang, Xuejian and Yu, Lantao and Ren, Kan and Tao, Guanyu and Zhang, Weinan and Yu, Yong and Wang, Jun},
  booktitle={Proceedings of the 23rd ACM SIGKDD International Conference on Knowledge Discovery and Data Mining},
  pages={2051--2059},
  year={2017},
  organization={ACM},
  annote={In this article the researchers utilize deep learning to learn the relationship between articles and there meta data in conjunction with a deep learning model to learn the underlying preference of the article reviewers which cannot be predicted using bag of word techniques since
  article reviewer preferences depends more on the writing quality vs. specific keywords found in the metadata or the article. This specific approach was focused on surfacing a subset articles for content reviewers for media companines like Buzzfeed to reduces their workload. These reviewers 
  then forward interesting/appropriate articles to machine learning models that will match these articles to consumers, an area that has drawn much more attention from researchers. The researchers touched on two traditional recommender techniques in their related work sections stating that collaborative
  filtering runs into issues with sparse datasets and thus faces the cold-start problem, and also touched on the content-based recommendation approach which deals with relationships between items in the dataset and won't run into issues with a cold start since they're reliant on the relationship between items, however,
  the content-based approach runs into issues with providing personalized reccomendations to the user, a crucial component of an effective recommendation system. The focus of the researchers was to create models that learned from the textual content of the articles as well as the articles' metadata like the author, original media,
  author city, and various other contextual information to provide relevant predictions. The authors' problem is unique to other reccommender systems in that the data is sparse becuase news articles become outdated quickly and thus the data they need to perform recommendations on won't have a lot of user interaction, rendering collaborative filtering
  essentially useless. Encouraging the researchers to focus on a content based recommendation system. For their CNN the researchers used character level neural network language models (NNLM) for low-level textual feature learning. The researchers found that the deep learning CNN technique had similar but better results that a logistic regression technique.
  }
}


@inproceedings{wu2016personal,
  title={Personal recommendation using deep recurrent neural networks in NetEase},
  author={Wu, Sai and Ren, Weichao and Yu, Chengchao and Chen, Gang and Zhang, Dongxiang and Zhu, Jingbo},
  booktitle={2016 IEEE 32nd International Conference on Data Engineering (ICDE)},
  pages={1218--1229},
  year={2016},
  organization={IEEE},
  annote={In this experiment the researchers forged a feedforward neural network which simulated collaborative filtering (accepted user purchase history as input to generate recommendation) and a conventional RNN to generate realtime reccomendations based of the user's browsing session. The researchers explain that an RNN is a 
  suitable choice of architercture for their problem because an RNN with on inner layer can determine the order of page access and an RNN with multiple layers can determine the combination of page accesses. One of the keypoints that the researchers make concerning their use of deep learning vs. traditional collaborative filtering is that
  collaborative filtering is effective at making recommendations based off of previous user choices whereas their deep learning model is able to make a new prediction based off of the given information of a user. To train their model the researchers used Stochastic Gradient Descent. Something that was really interesting about the researchers'
  approach is that they would iteratively retrain the model in real-time to give the most up to date predictions of how users interacted with the ecosystem. Another aspect that was interesting about the researchers' approach was their use of a "history state" that they used to consilidate decisions that dated outside of their sliding window of 
  sessions concept. The researchers showed that their RNN approach significantly outperformed a collaborative filtering approach.
  }
}

%% HUGE related work here
@article{Kaplan2018,
  title={Striving to earn more: a survey of work strategies and tool use among crowd workers},
  author={Kaplan, Toni and Saito, Susumu and Hara, Kotaro and Bigham, Jeffrey P},
  booktitle={Sixth AAAI Conference on Human Computation and Crowdsourcing},
  year={2018},
  annote={This paper outlines survey results collected from 360 crowdsourcing workers on their approach for earning money on crowdsourcing platforms and the tools they used. One of the interesting tools that crowdsource workers use is called Preview and Accept (commonly referred to as PandA) which finds similar
  tasks and batches them together to decrease the time that workers spend finding tasks. The survey outlines the inequality between workers and requesters and ultimately how requesters have a way to judge workers based on the reputation, workers don't actually have a way to sift through and filter tasks to find tasks that are efficient and feasible. This is
  directly related to the problem that I am trying to solve, improving efficiency of workers in crowdsourcing platforms. The researchers explain that much of the previous work done on task recommendation in Amazon MTURK deals with information available in the AMT search like task keywords, reward, qualificatoin, in conjunction with the
  browser extension Turkopticon, a tool used by crowdworkers to rate requesters, to surface wage-efficient tasks. The researchers also discussed their perspective that task recommendation systems should focus on content dependent factors which affect the amount of time it takes to complete a task like the types of media in a task and the 
  amount of inputs outlined in the task. The survey results reported a large number of forums for MTURK crowdworkers which would be really helpful for empathizing common problems that crowdworkers experience. One of the interesting things from the survey was the level of frustration for finding tasks within the platform and that not being
  able to find anymore tasks to complete was a high motivation for ending a working session. This is a really important piece of feedback that provides validation for researching approaches for improving the efficiency of crowd workers. An interesting facet of the survey results was the reported priority of crowdworkers on selecting a task in which "Pay per HIT", "Expected Task Completion Time",
  and "Requester Reputation" were ranked in the order previously described. In the researchers' future work section they outlined several possibilities for applying machine learning to improve crowdworker efficiency. The researchers outlined some potential approaches: develop a machine learning algorithm to surface recommendations based off of task feasibility and the
  estimated time it would take to complete the task, as well as to use the task content and metadata to help make a prediction. }
}

%% MTURK recommendation system: (Hanrahan et al. 2015), (Alsayasneh et al. 2018)

@inproceedings{amer2016task,
  title={Task Composition in Crowdsourcing},
  author={Amer-Yahia, Sihem and Gaussier, Eric and Leroy, Vincent and Pilourdault, Julien and Borromeo, Ria and Toyama, Motomichi},
  year={2016}
}
%% Read Alan Turing's Computing Machinery and Intelligence

%% Find articles on Collaborative Topic Regression

%% Content based recommendation
@incollection{pazzani2007content,
  title={Content-based recommendation systems},
  author={Pazzani, Michael J and Billsus, Daniel},
  booktitle={The adaptive web},
  pages={325--341},
  year={2007},
  publisher={Springer}
}

@incollection{aggarwal2012survey,
  title={A survey of text classification algorithms},
  author={Aggarwal, Charu C and Zhai, ChengXiang},
  booktitle={Mining text data},
  pages={163--222},
  year={2012},
  publisher={Springer}
}

%% Survey on earnings
@inproceedings{hara2018data,
  title={A data-driven analysis of workers' earnings on amazon mechanical turk},
  author={Hara, Kotaro and Adams, Abigail and Milland, Kristy and Savage, Saiph and Callison-Burch, Chris and Bigham, Jeffrey P},
  booktitle={Proceedings of the 2018 CHI Conference on Human Factors in Computing Systems},
  pages={449},
  year={2018},
  organization={ACM}
}

% common types of HITs in MTURK
@inproceedings{difallah2015dynamics,
  title={The dynamics of micro-task crowdsourcing: The case of amazon mturk},
  author={Difallah, Djellel Eddine and Catasta, Michele and Demartini, Gianluca and Ipeirotis, Panagiotis G and Cudr{\'e}-Mauroux, Philippe},
  booktitle={Proceedings of the 24th international conference on world wide web},
  pages={238--247},
  year={2015},
  organization={International World Wide Web Conferences Steering Committee}
}

% Task recommendation in MTurk, matrix factorization
@inproceedings{yuen2012task,
  title={Task recommendation in crowdsourcing systems},
  author={Yuen, Man-Ching and King, Irwin and Leung, Kwong-Sak},
  booktitle={Proceedings of the first international workshop on crowdsourcing and data mining},
  pages={22--26},
  year={2012},
  organization={ACM}
}

% Task recommendation in MTURK, BoW and Classification
@inproceedings{ambati2011towards,
  title={Towards task recommendation in micro-task markets},
  author={Ambati, Vamsi and Vogel, Stephan and Carbonell, Jaime},
  booktitle={Workshops at the Twenty-Fifth AAAI Conference on Artificial Intelligence},
  year={2011}
}

%% Crowdsourcing is growing
@article{kuek2015global,
  title={The global opportunity in online outsourcing},
  author={Kuek, Siou Chew and Paradi-Guilford, Cecilia and Fayomi, Toks and Imaizumi, Saori and Ipeirotis, Panos and Pina, Patricia and Singh, Manpreet},
  year={2015},
  publisher={World Bank, Washington, DC}
}